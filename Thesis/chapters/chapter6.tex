%%%%%%%%%%%%%%%%%%%%%%%%%%%%%%%%%%%%%%%%%%%%%%%%%%%%%%%%%%%%%%%%%%%%%%%%%%%%%%%%
\section{Efekt końcowy}
%%%%%%%%%%%%%%%%%%%%%%%%%%%%%%%%%%%%%%%%%%%%%%%%%%%%%%%%%%%%%%%%%%%%%%%%%%%%%%%%

Efektem pracy jest system, zaimplementowany zgodnie z projektem, przedstawionym przy pomocy opisów i diagramów UML, spełniający wymagania postawione mu w rozdziale wprowadzającym pracy:

\begin{itemize}
    \item Aplikacja kliencka działa w przeglądarce internetowej, dlatego w odróżnieniu od rozwiązań istniejących na rynku, może zostać wykorzystana na wszystkich platformach posiadających to oprogramowanie, wliczając w to platformy mobilne.
    \item Aplikacja pojedynczej strony jest w pełni responsywna, dzięki zastosowanym arkuszom CSS.
    \item System korzysta z różnych rodzajów mechanizmów autoryzacji, zapewniających bezpieczeństwo danych użytkownika oraz wyświetlanie adekwatnych treści po stronie klienta.
    \item Aplikacja serwerowa wykorzystuje kilka warstw logicznych, luźno powiązanych ze sobą klas i wzorzec zastrzyków zależności, pozwalających na łatwą modyfikację i rozszerzenie projektu.
\end{itemize}

%%%%%%%%%%%%%%%%%%%%%%%%%%%%%%%%%%%%%%%%%%%%%%%%%%%%%%%%%%%%%%%%%%%%%%%%%%%%%%%%
\section{Dalszy rozwój aplikacji}
%%%%%%%%%%%%%%%%%%%%%%%%%%%%%%%%%%%%%%%%%%%%%%%%%%%%%%%%%%%%%%%%%%%%%%%%%%%%%%%%

W przypadku dalszego rozwoju aplikacji należałoby rozważyć następujące rozszerzenia:

\begin{itemize}
    \item Wprowadzenie ról użytkowników takich jak: administrator systemu oraz zarządca i członkowie jednostki, jako użytkownicy systemu, posiadający odrębne możliwości ułatwiające ich współpracę.
    \item Dodanie opcji generowania raportów i dokumentów do pobrania ze strony, przydatnych m.in. przy załatwianiu spraw urzędowych.
    \item Przyznawanie odznak i rang członkom jednostek, rozbudowujących ich uprawnienia w systemie.
    \item Rozbudowanie systemu o nowe moduły np. rozliczania czasu pracy.
    \item Przebudowanie systemu w sieć biznesową, łączącą wiele jednostek na mapie Polski, uwzględniając transfery sprzętu i członków zespołu między nimi.
\end{itemize}

%%%%%%%%%%%%%%%%%%%%%%%%%%%%%%%%%%%%%%%%%%%%%%%%%%%%%%%%%%%%%%%%%%%%%%%%%%%%%%%%
\section{Wnioski po ukończeniu pracy}
%%%%%%%%%%%%%%%%%%%%%%%%%%%%%%%%%%%%%%%%%%%%%%%%%%%%%%%%%%%%%%%%%%%%%%%%%%%%%%%%

W trakcie tworzenia pracy, jej początkowe założenia uległy kilku modyfikacjom, które wpłynęły na jej ostateczną formę i wartość merytoryczną. Początkowo, praca zakładała opisanie wprowadzania zmian w systemie, w kolejnych iteracjach projektu, we współpracy z użytkownikami.

Niestety, ze względu na obostrzenia wprowadzone w związku z sytuacją pandemiczną SARS-CoV-2, możliwości nawiązania bezpośredniej współpracy z jednostkami Ochotniczej Straży Pożarnej, zostały ograniczone. W związku z tym, w pracy kompleksowo przedstawiono założenia i sposób działania oraz rozwoju współczesnych systemów internetowych.

W ramach pracy wykorzystano, uporządkowano i rozwinięto wiedzę inżynierską, nabytą w trakcie trwania studiów, z wyszczególnieniem kursów:

\begin{itemize}
    \item Nowoczesne Technologie Internetowe - poświęconego tworzeniu aplikacji internetowych z użyciem pełnych stosów technologicznych.
    \item Inżynieria Oprogramowania - tworzącego wprowadzenie do tematu wzorców projektowych oraz metod dokumentowania projektów przy pomocy diagramów UML.
    \item Podstawy Sieci Komputerowych - dającego możliwość własnoręcznego wykorzystania protokołów tworzących sieć Internet.
\end{itemize}

Utworzenie prostego, lecz kompleksowego, rozwiązania biznesowego było dla autora okazją, do rewizji teoretycznej wiedzy o technologiach związanych z platformą internetową, wykorzystywaną w jego pracy zawodowej. Dodatkowo, praca umożliwiła przetestowanie eksperymentalnej platformy Blazor, mogącej wyznaczyć drogę dla nowej generacji platform aplikacji klienckich. 

Otwarta architektura Internetu, przyczyniła się do jego prężnego rozwoju w ostatnich dekadach. Bazując na ideach omawianych i standaryzowanych, przez inżynierów oprogramowania z całego świata, przeszedł drogę od niezawodnego sposobu przesyłania danych, do najpopularniejszej na świecie platformy, dostarczającej użytkownikom aplikacje wykorzystywane w ich codziennym życiu. Technologie internetowe zawdzięczają swoją obecną formę społeczności, w której udział poszczególnych twórców oprogramowania, ma wpływ na ich dalszy rozwój, skupiony na ciągłym doskonaleniu doświadczeń zarówno użytkowników, jak i twórców systemów. %ufff

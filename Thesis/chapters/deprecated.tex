%%%%%%%%%%%%%%%%%%%%%%%%%%%%%%%%%%%%%%%%%%%%%%%%%%%%%%%%%%%%%%%%%%%%%%%%%%%%%%%%
Niepotrzebny w kontekście końcowego efektu fragment o zaletach PWA ze wstępu
%%%%%%%%%%%%%%%%%%%%%%%%%%%%%%%%%%%%%%%%%%%%%%%%%%%%%%%%%%%%%%%%%%%%%%%%%%%%%%%%

Aplikacje PWA zostały docenione przez wielu przedstawicieli firm funkcjonujących na rynku IT i są chętnie opisywane w ramach blogów pracowniczych \cite{Moscibroda.PWA-przyszlosc-aplikacji-natywnych}. Trudno się temu dziwić, z uwagi na fakt iż dzięki swym zaletom rozwiązują kilka problemów dotykających współczesny rynek oprogramowania:

\begin{itemize}
    \item Stają się niezawodne, niezależnie od szybkości i stanu sieci internetowej, dzięki wcześniejszemu pobieraniu danych na urządzenie. Pozwala to również na zaoszczędzenie do 92\% transferu danych w taryfie sieciowej użytkowników \cite{Warcholinski.PWA-ROI}.
    \item Zapewniają szybkie i skuteczne działanie, które wywiera istotny wpływ na zainteresowanie narzędziem ze strony klientów. Przyczynia się do tego nawet dziesięciokrotny wzrost szybkości ładowania elementów PWA, względem ich przeglądarkowych odpowiedników.
    \item Wykorzystanie ich określa się jako bardziej angażujące, ze względu na ulepszony mechanizm ich uruchamiania, rozbudowanej kontroli wyświetlania i dodatkom takim jak powiadomienia systemowe.
    \item Potencjalnemu użytkownikowi łatwiej jest przetestować narzędzie przy pomocy przeglądarki internetowej i zdecydować czy zamierza dalej z niego korzystać. Czyni to PWA bardzo atrakcyjnymi produktami w sytuacji w, której ponad 51\% posiadaczy smartfonów jest nieprzychylna instalacji nowych aplikacji na swoich urządzeniach \cite{Warcholinski.PWA-ROI}.
\end{itemize}

%%%%%%%%%%%%%%%%%%%%%%%%%%%%%%%%%%%%%%%%%%%%%%%%%%%%%%%%%%%%%%%%%%%%%%%%%%%%%%%%
PWA is no more
%%%%%%%%%%%%%%%%%%%%%%%%%%%%%%%%%%%%%%%%%%%%%%%%%%%%%%%%%%%%%%%%%%%%%%%%%%%%%%%%

    \item Obowiązkowym punktem pracy powinna stać się również analiza kompatybilnego z wcześniej wybranym stosem technologicznym rozwiązania Progressive Web Application (PWA). Dzięki niemu tworzona aplikacja stanie się o wiele bardziej wydajnym i komfortowym w obsłudze produktem.

%%%%%%%%%%%%%%%%%%%%%%%%%%%%%%%%%%%%%%%%%%%%%%%%%%%%%%%%%%%%%%%%%%%%%%%%%%%%%%%%
Mdły i drętwy przegląd literatury...
%%%%%%%%%%%%%%%%%%%%%%%%%%%%%%%%%%%%%%%%%%%%%%%%%%%%%%%%%%%%%%%%%%%%%%%%%%%%%%%%

Wskutek dużego wzrostu popularności internetowych aplikacji klienckich i serwerowych, na przełomie ostatnich dwóch dekad, w ofercie wydawniczej pojawiło się wiele pozycji z nimi związanych. 

W publicystyce branżowej odnaleźć można pozycje traktujące o zasadności użycia aplikacji internetowych, jako rozwiązania problemu dostarczenia wieloplatformowego oprogramowania do użytkowników. Zdaniem autorów \cite{Latif2016} zaletą takiego podejścia jest fakt zunifikowanego procesu dostosowywania oprogramowania do wielkości ekranów czy unikalnych parametrów urządzeń. Jednakże wskazują oni na możliwe problemy związane z obsługą urządzeń zewnętrznych oraz bezpośrednie przełożenie wydajności przeglądarki internetowej na jakość doświadczenia z aplikacją. Na szczęście zgodnie z \cite{SmeetsRuben2016TiWB} technologie internetowe rozwijają się bardzo prężnie coraz lepiej rozwiązując te i kolejne problemy postawione przez rynek IT. Ponadto, twórcy artykułu \cite{Charland2011} zapewniają, że oprogramowanie internetowe jest tworzone i dystrybuowane zdecydowanie szybciej, a także taniej w porównaniu do aplikacji natywnych, z czym trudno się nie zgodzić. 

Rozwijane są również metody przyspieszenia pracy z aplikacjami internetowymi tak jak robią to twórcy platform programistycznych opisanych w \cite{ReactNative} czy \cite{NativeScript}. Szczegółowe porównanie popularnych w ostatnich latach strategii zamieszczono w \cite{Biørn-Hansen2017}, a ostatecznie wszystkie z nich, łączy minimalna ilość modyfikacji potrzebnych by ze standardowego programu internetowego, otrzymać program będący równie wydajny co aplikacja natywna.

Mimo szybko zmieniających się trendów i dużego postępu na rynku systemów internetowych, dużą część pozycji książkowych wydanych po roku 2010 można wciąż uznać za aktualne m.in. dzięki konsekwentnym etapom rozwijania standardów panujących w technologiach internetowych \cite{W3C}.

Na potrzebę zgłębienia tematu spopularyzowanych ostatnio aplikacji PWA odpowiedział między innymi autor \cite{Tal2017} opisując nie tylko proces modyfikacji oprogramowania, ale również metody pozwalające  na zachęcenie użytkownika do skorzystania z  takich aplikacji.

Nie mogło również zabraknąć pozycji związanych z bardzo ważnym aspektem dostosowywania systemu do wymagań ich potencjalnego użytkownika. Zdaniem autora \cite{Brian2009} w procesie projektowania interfejsu i funkcjonalności aplikacji klienckich można opisać wiele standardowych schematów, a nawet nakreślić konkretny zarys architektury związanej z rozwiązaniami komunikacji z użytkownikiem na urządzeniach mobilnych. Jest to szczególnie ważne spostrzeżenie, ze względu na fakt ogromnej bazy użytkowników takich właśnie urządzeń.

Pozycje \cite{Mike2003} i \cite{Jeff2013} to zbiory metodologii badania oczekiwań klienta wobec produktu, który chcemy mu sprzedać, jak i procesów rozwoju “doświadczenia użytkownika” (User Experience). Procesy te nie są związane bezpośrednio z dziedziną inżynierii oprogramowania, jednakże stanowią niezbędny suplement w procesie tworzenia wysokiej jakości systemów.

Kolejną tematyką, którą chciałbym omówić w ramach przeglądu literatury są aplikacje serwerowe. W związku z większym zróżnicowaniem implementacji rozwiązań problemów trudniej opisać jest ich aspekty w sposób tak ogólny jak technologie aplikacji klienckich. 

Autor \cite{Gaurav2018} uważa iż bardzo dobrym wyborem do zrealizowania swoich aplikacji serwerowych jest platforma ASP.NET Core. Zaznacza, że platforma ta wspólnie z wykorzystaniem architektury REST API \cite{Margaret2020} pozwala na stworzenie przejrzystego i taniego w obliczeniach oprogramowania, które łatwo można rozbudowywać dzięki modularności rozwiązania. Jednym z najważniejszych modułów opisanych w publikacji \cite{Holger2018} pozostaje natomiast Entity Framework Core stanowiący narzędzie do mapowania obiektowo-relacyjnego. Rozwiązanie komunikacji z bazą danych przy jego pomocy nie tylko czyni sam proces bardziej zrozumiały i skalowalny dla programisty, ale również przynosi korzyści w aspektach bezpieczeństwa.

Podsumowując, istnieje duża liczba literatury z zakresu internetowych systemów informatycznych. Na potrzeby niniejszej pracy przeanalizowałem zbiór kilku wartościowych pozycji. Jednakże należy podkreślić, że trwająca ciągle rywalizacja między twórcami platform programistycznych i bibliotek jest w stanie w bardzo krótkim czasie modyfikować trendy ich eksploatowania.

%%%%%%%%%%%%%%%%%%%%%%%%%%%%%%%%%%%%%%%%%%%%%%%%%%%%%%%%%%%%%%%%%%%%%%%%%%%%%%%%
Duża tabela scenariusza przypadku użycia... za duża
%%%%%%%%%%%%%%%%%%%%%%%%%%%%%%%%%%%%%%%%%%%%%%%%%%%%%%%%%%%%%%%%%%%%%%%%%%%%%%%%

\iffalse
\begin{table}[]
    \centering
    \begin{adjustbox} {totalheight=\textheight}
        \begin{tabularx}{\textwidth}{|S|L|}
            \hline
            Nazwa & Dodawanie zasobu użytkownika \\
            \hline
            Aktorzy & Użytkownik uwierzytelniony \\
            \hline
            Krótki opis & Użytkownik uwierzytelniony dodaje przypisaną do niego encje zasobu w systemie \\
            \hline
            Warunki wstępne & 
            Użytkownik korzysta z widoku tworzenia nowego zasobu i posiada zapisany w pamięci swojej przeglądarki internetowej token uwierzytelniający uprawniający go do tworzenia przypisanych do niego zasobów \\
            \hline
            Warunki końcowe & 
            Zasób przypisany do użytkownika jest zapisany w bazie danych systemu, a aplikacja kliencka komunikuje o tym użytkownika \\
            \hline
            Główny przepływ zdarzeń & 
            \begin{enumerate}
                \item
            \end{enumerate} \\
            \hline
            Opcjonalny przepływ zdarzeń &
            \begin{enumerate}
                \item[1a.]
            \end{enumerate} \\
            \hline
            Alternatywny przepływ zdarzeń & 
            \begin{enumerate}
                \item[1a.]
            \end{enumerate} \\
            \hline
            Specjalne wymagania & 
            \begin{enumerate}
                \item
            \end{enumerate} \\
            \hline
        \end{tabularx}
    \end{adjustbox}
    \caption{Scenariusz dodawania zasobu użytkownika}
    \label{tab:scenario.create-resource}
\end{table}
\fi